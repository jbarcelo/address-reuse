
%% bare_jrnl.tex
%% V1.3
%% 2007/01/11
%% by Michael Shell
%% see http://www.michaelshell.org/
%% for current contact information.
%%
%% This is a skeleton file demonstrating the use of IEEEtran.cls
%% (requires IEEEtran.cls version 1.7 or later) with an IEEE journal paper.
%%
%% Support sites:
%% http://www.michaelshell.org/tex/ieeetran/
%% http://www.ctan.org/tex-archive/macros/latex/contrib/IEEEtran/
%% and
%% http://www.ieee.org/



% *** Authors should verify (and, if needed, correct) their LaTeX system  ***
% *** with the testflow diagnostic prior to trusting their LaTeX platform ***
% *** with production work. IEEE's font choices can trigger bugs that do  ***
% *** not appear when using other class files.                            ***
% The testflow support page is at:
% http://www.michaelshell.org/tex/testflow/


%%*************************************************************************
%% Legal Notice:
%% This code is offered as-is without any warranty either expressed or
%% implied; without even the implied warranty of MERCHANTABILITY or
%% FITNESS FOR A PARTICULAR PURPOSE!
%% User assumes all risk.
%% In no event shall IEEE or any contributor to this code be liable for
%% any damages or losses, including, but not limited to, incidental,
%% consequential, or any other damages, resulting from the use or misuse
%% of any information contained here.
%%
%% All comments are the opinions of their respective authors and are not
%% necessarily endorsed by the IEEE.
%%
%% This work is distributed under the LaTeX Project Public License (LPPL)
%% ( http://www.latex-project.org/ ) version 1.3, and may be freely used,
%% distributed and modified. A copy of the LPPL, version 1.3, is included
%% in the base LaTeX documentation of all distributions of LaTeX released
%% 2003/12/01 or later.
%% Retain all contribution notices and credits.
%% ** Modified files should be clearly indicated as such, including  **
%% ** renaming them and changing author support contact information. **
%%
%% File list of work: IEEEtran.cls, IEEEtran_HOWTO.pdf, bare_adv.tex,
%%                    bare_conf.tex, bare_jrnl.tex, bare_jrnl_compsoc.tex
%%*************************************************************************

% Note that the a4paper option is mainly intended so that authors in
% countries using A4 can easily print to A4 and see how their papers will
% look in print - the typesetting of the document will not typically be
% affected with changes in paper size (but the bottom and side margins will).
% Use the testflow package mentioned above to verify correct handling of
% both paper sizes by the user's LaTeX system.
%
% Also note that the "draftcls" or "draftclsnofoot", not "draft", option
% should be used if it is desired that the figures are to be displayed in
% draft mode.
%
\documentclass[journal]{IEEEtran}

\newtheorem{definition}{Definition}
\newtheorem{proposition}{Proposition}
\newtheorem{theorem}{Theorem}
\newtheorem{remark}{Remark}

% to add comments in different colors (added by Azadeh)
\usepackage[usenames]{color}
\usepackage{lscape}
\usepackage{soul}
\newcommand{\blue}[1]{{\color{blue} #1}}
\newcommand{\red}[1]{{\color{red} #1}}
\newcommand{\black}[1]{{\color{black} #1}}

\newcommand{\Az}[1]{{\color{blue}{#1}}}
%\newcommand{\AzCom}[1]{{\it \color{magenta} [#1]}}
\newcommand{\AzCom}[1]{}
%\newcommand{\AzDel}[1]{{\color{red} \st{#1}}}
%\newcommand{\AzDel}[1]{{\color{Gray}{\{#1\}}}}
\newcommand{\AzDel}[1]{}

\newdimen\snellbaselineskip
\newdimen\snellskip
\snellskip=1.5ex
\snellbaselineskip=\baselineskip
\def\srule{\omit\kern.5em\vrule\kern-.5em}
\newbox\bigstrutbox
\setbox\bigstrutbox=\hbox{\vrule height14.5pt depth9.5pt width0pt}
\def\bigstrut{\relax\ifmmode\copy\bigstrutbox\else\unhcopy\bigstrutbox\fi}
\def\middlehrule#1#2{\noalign{\kern-\snellbaselineskip\kern\snellskip}
&\multispan#1\strut\hrulefill
&\omit\hbox to.5em{\hrulefill}\vrule
height \snellskip\kern-.5em&\multispan#2\hrulefill\cr}


\makeatletter
\def\bordermatrix#1{\begingroup \m@th
  \@tempdima 8.75\p@
  \setbox\z@\vbox{%
    \def\cr{\crcr\noalign{\kern2\p@\global\let\cr\endline}}%
    \ialign{$##$\hfil\kern2\p@\kern\@tempdima&\thinspace\hfil$##$\hfil
      &&\quad\hfil$##$\hfil\crcr
      \omit\strut\hfil\crcr\noalign{\kern-\snellbaselineskip}%
      #1\crcr\omit\strut\cr}}%
  \setbox\tw@\vbox{\unvcopy\z@\global\setbox\@ne\lastbox}%
  \setbox\tw@\hbox{\unhbox\@ne\unskip\global\setbox\@ne\lastbox}%
  \setbox\tw@\hbox{$\kern\wd\@ne\kern-\@tempdima\left(\kern-\wd\@ne
    \global\setbox\@ne\vbox{\box\@ne\kern2\p@}%
    \vcenter{\kern-\ht\@ne\unvbox\z@\kern-\snellbaselineskip}\,\right)$}%
  \null\;\vbox{\kern\ht\@ne\box\tw@}\endgroup}
\makeatletter

\makeatletter
\def\bordermatrix#1{\begingroup \m@th
  \@tempdima 8.75\p@
  \setbox\z@\vbox{%
    \def\cr{\crcr\noalign{\kern2\p@\global\let\cr\endline}}%
    \ialign{$##$\hfil\kern2\p@\kern\@tempdima&\thinspace\hfil$##$\hfil
      &&\quad\hfil$##$\hfil\crcr
      \omit\strut\hfil\crcr\noalign{\kern-\snellbaselineskip}%
      #1\crcr\omit\strut\cr}}%
  \setbox\tw@\vbox{\unvcopy\z@\global\setbox\@ne\lastbox}%
  \setbox\tw@\hbox{\unhbox\@ne\unskip\global\setbox\@ne\lastbox}%
  \setbox\tw@\hbox{$\kern\wd\@ne\kern-\@tempdima\left(\kern-\wd\@ne
    \global\setbox\@ne\vbox{\box\@ne\kern2\p@}%
    \vcenter{\kern-\ht\@ne\unvbox\z@\kern-\snellbaselineskip}\,\right)$}%
  \null\;\vbox{\kern\ht\@ne\box\tw@}\endgroup}
\makeatletter

%
% If IEEEtran.cls has not been installed into the LaTeX system files,
% manually specify the path to it like:
% \documentclass[journal]{../sty/IEEEtran}





% Some very useful LaTeX packages include:
% (uncomment the ones you want to load)


% *** MISC UTILITY PACKAGES ***
%
%\usepackage{ifpdf}
% Heiko Oberdiek's ifpdf.sty is very useful if you need conditional
% compilation based on whether the output is pdf or dvi.
% usage:
% \ifpdf
%   % pdf code
% \else
%   % dvi code
% \fi
% The latest version of ifpdf.sty can be obtained from:
% http://www.ctan.org/tex-archive/macros/latex/contrib/oberdiek/
% Also, note that IEEEtran.cls V1.7 and later provides a builtin
% \ifCLASSINFOpdf conditional that works the same way.
% When switching from latex to pdflatex and vice-versa, the compiler may
% have to be run twice to clear warning/error messages.






% *** CITATION PACKAGES ***
%
\usepackage{cite}
% cite.sty was written by Donald Arseneau
% V1.6 and later of IEEEtran pre-defines the format of the cite.sty package
% \cite{} output to follow that of IEEE. Loading the cite package will
% result in citation numbers being automatically sorted and properly
% "compressed/ranged". e.g., [1], [9], [2], [7], [5], [6] without using
% cite.sty will become [1], [2], [5]--[7], [9] using cite.sty. cite.sty's
% \cite will automatically add leading space, if needed. Use cite.sty's
% noadjust option (cite.sty V3.8 and later) if you want to turn this off.
% cite.sty is already installed on most LaTeX systems. Be sure and use
% version 4.0 (2003-05-27) and later if using hyperref.sty. cite.sty does
% not currently provide for hyperlinked citations.
% The latest version can be obtained at:
% http://www.ctan.org/tex-archive/macros/latex/contrib/cite/
% The documentation is contained in the cite.sty file itself.






% *** GRAPHICS RELATED PACKAGES ***
%
\ifCLASSINFOpdf
  % \usepackage[pdftex]{graphicx}
  % declare the path(s) where your graphic files are
  % \graphicspath{{../pdf/}{../jpeg/}}
  % and their extensions so you won't have to specify these with
  % every instance of \includegraphics
  % \DeclareGraphicsExtensions{.pdf,.jpeg,.png}
\else
  % or other class option (dvipsone, dvipdf, if not using dvips). graphicx
  % will default to the driver specified in the system graphics.cfg if no
  % driver is specified.
  \usepackage[dvips]{graphicx}
  % declare the path(s) where your graphic files are
  % \graphicspath{{../eps/}}
  % and their extensions so you won't have to specify these with
  % every instance of \includegraphics
  \DeclareGraphicsExtensions{.eps}
\fi
% graphicx was written by David Carlisle and Sebastian Rahtz. It is
% required if you want graphics, photos, etc. graphicx.sty is already
% installed on most LaTeX systems. The latest version and documentation can
% be obtained at:
% http://www.ctan.org/tex-archive/macros/latex/required/graphics/
% Another good source of documentation is "Using Imported Graphics in
% LaTeX2e" by Keith Reckdahl which can be found as epslatex.ps or
% epslatex.pdf at: http://www.ctan.org/tex-archive/info/
%
% latex, and pdflatex in dvi mode, support graphics in encapsulated
% postscript (.eps) format. pdflatex in pdf mode supports graphics
% in .pdf, .jpeg, .png and .mps (metapost) formats. Users should ensure
% that all non-photo figures use a vector format (.eps, .pdf, .mps) and
% not a bitmapped formats (.jpeg, .png). IEEE frowns on bitmapped formats
% which can result in "jaggedy"/blurry rendering of lines and letters as
% well as large increases in file sizes.
%
% You can find documentation about the pdfTeX application at:
% http://www.tug.org/applications/pdftex





% *** MATH PACKAGES ***
%
\usepackage[cmex10]{amsmath}
% A popular package from the American Mathematical Society that provides
% many useful and powerful commands for dealing with mathematics. If using
% it, be sure to load this package with the cmex10 option to ensure that
% only type 1 fonts will utilized at all point sizes. Without this option,
% it is possible that some math symbols, particularly those within
% footnotes, will be rendered in bitmap form which will result in a
% document that can not be IEEE Xplore compliant!
%
% Also, note that the amsmath package sets \interdisplaylinepenalty to 10000
% thus preventing page breaks from occurring within multiline equations. Use:
%\interdisplaylinepenalty=2500
% after loading amsmath to restore such page breaks as IEEEtran.cls normally
% does. amsmath.sty is already installed on most LaTeX systems. The latest
% version and documentation can be obtained at:
% http://www.ctan.org/tex-archive/macros/latex/required/amslatex/math/

\usepackage{amssymb}
\usepackage{amsfonts}




% *** SPECIALIZED LIST PACKAGES ***
%
%\usepackage{algorithmic}
% algorithmic.sty was written by Peter Williams and Rogerio Brito.
% This package provides an algorithmic environment fo describing algorithms.
% You can use the algorithmic environment in-text or within a figure
% environment to provide for a floating algorithm. Do NOT use the algorithm
% floating environment provided by algorithm.sty (by the same authors) or
% algorithm2e.sty (by Christophe Fiorio) as IEEE does not use dedicated
% algorithm float types and packages that provide these will not provide
% correct IEEE style captions. The latest version and documentation of
% algorithmic.sty can be obtained at:
% http://www.ctan.org/tex-archive/macros/latex/contrib/algorithms/
% There is also a support site at:
% http://algorithms.berlios.de/index.html
% Also of interest may be the (relatively newer and more customizable)
% algorithmicx.sty package by Szasz Janos:
% http://www.ctan.org/tex-archive/macros/latex/contrib/algorithmicx/




% *** ALIGNMENT PACKAGES ***
%
%\usepackage{array}
% Frank Mittelbach's and David Carlisle's array.sty patches and improves
% the standard LaTeX2e array and tabular environments to provide better
% appearance and additional user controls. As the default LaTeX2e table
% generation code is lacking to the point of almost being broken with
% respect to the quality of the end results, all users are strongly
% advised to use an enhanced (at the very least that provided by array.sty)
% set of table tools. array.sty is already installed on most systems. The
% latest version and documentation can be obtained at:
% http://www.ctan.org/tex-archive/macros/latex/required/tools/


%\usepackage{mdwmath}
%\usepackage{mdwtab}
% Also highly recommended is Mark Wooding's extremely powerful MDW tools,
% especially mdwmath.sty and mdwtab.sty which are used to format equations
% and tables, respectively. The MDWtools set is already installed on most
% LaTeX systems. The lastest version and documentation is available at:
% http://www.ctan.org/tex-archive/macros/latex/contrib/mdwtools/


% IEEEtran contains the IEEEeqnarray family of commands that can be used to
% generate multiline equations as well as matrices, tables, etc., of high
% quality.


%\usepackage{eqparbox}
% Also of notable interest is Scott Pakin's eqparbox package for creating
% (automatically sized) equal width boxes - aka "natural width parboxes".
% Available at:
% http://www.ctan.org/tex-archive/macros/latex/contrib/eqparbox/





% *** SUBFIGURE PACKAGES ***
%\usepackage[tight,footnotesize]{subfigure}
% subfigure.sty was written by Steven Douglas Cochran. This package makes it
% easy to put subfigures in your figures. e.g., "Figure 1a and 1b". For IEEE
% work, it is a good idea to load it with the tight package option to reduce
% the amount of white space around the subfigures. subfigure.sty is already
% installed on most LaTeX systems. The latest version and documentation can
% be obtained at:
% http://www.ctan.org/tex-archive/obsolete/macros/latex/contrib/subfigure/
% subfigure.sty has been superceeded by subfig.sty.



%\usepackage[caption=false]{caption}
%\usepackage[font=footnotesize]{subfig}
% subfig.sty, also written by Steven Douglas Cochran, is the modern
% replacement for subfigure.sty. However, subfig.sty requires and
% automatically loads Axel Sommerfeldt's caption.sty which will override
% IEEEtran.cls handling of captions and this will result in nonIEEE style
% figure/table captions. To prevent this problem, be sure and preload
% caption.sty with its "caption=false" package option. This is will preserve
% IEEEtran.cls handing of captions. Version 1.3 (2005/06/28) and later
% (recommended due to many improvements over 1.2) of subfig.sty supports
% the caption=false option directly:
%\usepackage[caption=false,font=footnotesize]{subfig}
%
% The latest version and documentation can be obtained at:
% http://www.ctan.org/tex-archive/macros/latex/contrib/subfig/
% The latest version and documentation of caption.sty can be obtained at:
% http://www.ctan.org/tex-archive/macros/latex/contrib/caption/




% *** FLOAT PACKAGES ***
%
%\usepackage{fixltx2e}
% fixltx2e, the successor to the earlier fix2col.sty, was written by
% Frank Mittelbach and David Carlisle. This package corrects a few problems
% in the LaTeX2e kernel, the most notable of which is that in current
% LaTeX2e releases, the ordering of single and double column floats is not
% guaranteed to be preserved. Thus, an unpatched LaTeX2e can allow a
% single column figure to be placed prior to an earlier double column
% figure. The latest version and documentation can be found at:
% http://www.ctan.org/tex-archive/macros/latex/base/



%\usepackage{stfloats}
% stfloats.sty was written by Sigitas Tolusis. This package gives LaTeX2e
% the ability to do double column floats at the bottom of the page as well
% as the top. (e.g., "\begin{figure*}[!b]" is not normally possible in
% LaTeX2e). It also provides a command:
%\fnbelowfloat
% to enable the placement of footnotes below bottom floats (the standard
% LaTeX2e kernel puts them above bottom floats). This is an invasive package
% which rewrites many portions of the LaTeX2e float routines. It may not work
% with other packages that modify the LaTeX2e float routines. The latest
% version and documentation can be obtained at:
% http://www.ctan.org/tex-archive/macros/latex/contrib/sttools/
% Documentation is contained in the stfloats.sty comments as well as in the
% presfull.pdf file. Do not use the stfloats baselinefloat ability as IEEE
% does not allow \baselineskip to stretch. Authors submitting work to the
% IEEE should note that IEEE rarely uses double column equations and
% that authors should try to avoid such use. Do not be tempted to use the
% cuted.sty or midfloat.sty packages (also by Sigitas Tolusis) as IEEE does
% not format its papers in such ways.


%\ifCLASSOPTIONcaptionsoff
%  \usepackage[nomarkers]{endfloat}
% \let\MYoriglatexcaption\caption
% \renewcommand{\caption}[2][\relax]{\MYoriglatexcaption[#2]{#2}}
%\fi
% endfloat.sty was written by James Darrell McCauley and Jeff Goldberg.
% This package may be useful when used in conjunction with IEEEtran.cls'
% captionsoff option. Some IEEE journals/societies require that submissions
% have lists of figures/tables at the end of the paper and that
% figures/tables without any captions are placed on a page by themselves at
% the end of the document. If needed, the draftcls IEEEtran class option or
% \CLASSINPUTbaselinestretch interface can be used to increase the line
% spacing as well. Be sure and use the nomarkers option of endfloat to
% prevent endfloat from "marking" where the figures would have been placed
% in the text. The two hack lines of code above are a slight modification of
% that suggested by in the endfloat docs (section 8.3.1) to ensure that
% the full captions always appear in the list of figures/tables - even if
% the user used the short optional argument of \caption[]{}.
% IEEE papers do not typically make use of \caption[]'s optional argument,
% so this should not be an issue. A similar trick can be used to disable
% captions of packages such as subfig.sty that lack options to turn off
% the subcaptions:
% For subfig.sty:
% \let\MYorigsubfloat\subfloat
% \renewcommand{\subfloat}[2][\relax]{\MYorigsubfloat[]{#2}}
% For subfigure.sty:
% \let\MYorigsubfigure\subfigure
% \renewcommand{\subfigure}[2][\relax]{\MYorigsubfigure[]{#2}}
% However, the above trick will not work if both optional arguments of
% the \subfloat/subfig command are used. Furthermore, there needs to be a
% description of each subfigure *somewhere* and endfloat does not add
% subfigure captions to its list of figures. Thus, the best approach is to
% avoid the use of subfigure captions (many IEEE journals avoid them anyway)
% and instead reference/explain all the subfigures within the main caption.
% The latest version of endfloat.sty and its documentation can obtained at:
% http://www.ctan.org/tex-archive/macros/latex/contrib/endfloat/
%
% The IEEEtran \ifCLASSOPTIONcaptionsoff conditional can also be used
% later in the document, say, to conditionally put the References on a
% page by themselves.





% *** PDF, URL AND HYPERLINK PACKAGES ***
%
\usepackage{url}
% url.sty was written by Donald Arseneau. It provides better support for
% handling and breaking URLs. url.sty is already installed on most LaTeX
% systems. The latest version can be obtained at:
% http://www.ctan.org/tex-archive/macros/latex/contrib/misc/
% Read the url.sty source comments for usage information. Basically,
% \url{my_url_here}.

\usepackage{psfrag}



% *** Do not adjust lengths that control margins, column widths, etc. ***
% *** Do not use packages that alter fonts (such as pslatex).         ***
% There should be no need to do such things with IEEEtran.cls V1.6 and later.
% (Unless specifically asked to do so by the journal or conference you plan
% to submit to, of course. )


% correct bad hyphenation here

\def\Abf{{\mathbf{A}}}
\def\Pbf{{\mathbf{P}}}
\newcommand{\pr}[1]{\Pr \left\{#1\right\}}
\hyphenation{op-tical net-works semi-conduc-tor}


\begin{document}
%
% paper title
% can use linebreaks \\ within to get better formatting as desired
\title{Address Reuse in the Bitcoin Network}
%
%
% author names and IEEE memberships
% note positions of commas and nonbreaking spaces ( ~ ) LaTeX will not break
% a structure at a ~ so this keeps an author's name from being broken across
% two lines.
% use \thanks{} to gain access to the first footnote area
% a separate \thanks must be used for each paragraph as LaTeX2e's \thanks
% was not built to handle multiple paragraphs
%

\author{Name~Surname, %~\IEEEmembership{Member,~IEEE,}
        Name~Surname, %~\IEEEmembership{Fellow,~OSA,}
        Name~Surname, \IEEEmembership{Member, IEEE,}
        Name~Surname, %~\IEEEmembership{Fellow,~OSA,}
        and~Name~Surname,~\IEEEmembership{Senior~Member,~IEEE}% <-this % stops a space
\thanks{The authors are with Universitat Pompeu Fabra.
Roc Boronat 138, 08018 Barcelona, Catalunya, Spain.
E-mail: jaume.barcelo@upf.edu
This paper has been submitted to an IEEE journal.
}
}

% note the % following the last \IEEEmembership and also \thanks -
% these prevent an unwanted space from occurring between the last author name
% and the end of the author line. i.e., if you had this:
%
% \author{....lastname \thanks{...} \thanks{...} }
%                     ^------------^------------^----Do not want these spaces!
%
% a space would be appended to the last name and could cause every name on that
% line to be shifted left slightly. This is one of those "LaTeX things". For
% instance, "\textbf{A} \textbf{B}" will typeset as "A B" not "AB". To get
% "AB" then you have to do: "\textbf{A}\textbf{B}"
% \thanks is no different in this regard, so shield the last } of each \thanks
% that ends a line with a % and do not let a space in before the next \thanks.
% Spaces after \IEEEmembership other than the last one are OK (and needed) as
% you are supposed to have spaces between the names. For what it is worth,
% this is a minor point as most people would not even notice if the said evil
% space somehow managed to creep in.



% The paper headers
\markboth{Journal of \LaTeX\ Class Files,~Vol.~6, No.~1, January~2007}%
{Shell \MakeLowercase{\textit{et al.}}: Bare Demo of IEEEtran.cls for Journals}
% The only time the second header will appear is for the odd numbered pages
% after the title page when using the twoside option.
%
% *** Note that you probably will NOT want to include the author's ***
% *** name in the headers of peer review papers.                   ***
% You can use \ifCLASSOPTIONpeerreview for conditional compilation here if
% you desire.




% If you want to put a publisher's ID mark on the page you can do it like
% this:
%\IEEEpubid{0000--0000/00\$00.00~\copyright~2007 IEEE}
% Remember, if you use this you must call \IEEEpubidadjcol in the second
% column for its text to clear the IEEEpubid mark.



% use for special paper notices
%\IEEEspecialpapernotice{(Invited Paper)}




% make the title area
\maketitle


\begin{abstract}
Bitcoin is a peer-to-peer electronic cash system that maintains a public ledger with all transactions.
The public availability of this information has implications for the privacy of the users.
The public ledger consists of transactions that transfer funds from a set of inputs to a set of output addresses.
As long as those addresses cannot be linked to their owners, privacy is preserved.
The linking of addresses to owners results in privacy leaks.
The possibilities of linking addresses to owners are multiplied when addresses are used to receive funds more than once.
In this work we privacy-leaking effects of address reuse and gather statistics of address reuse in the Bitcoin network.
\end{abstract}
% IEEEtran.cls defaults to using nonbold math in the Abstract.
% This preserves the distinction between vectors and scalars. However,
% if the journal you are submitting to favors bold math in the abstract,
% then you can use LaTeX's standard command \boldmath at the very start
% of the abstract to achieve this. Many IEEE journals frown on math
% in the abstract anyway.

% Note that keywords are not normally used for peerreview papers.
\begin{IEEEkeywords}
Bitcoin, cryptocurrency, privacy, address reuse
\end{IEEEkeywords}






% For peer review papers, you can put extra information on the cover
% page as needed:
% \ifCLASSOPTIONpeerreview
% \begin{center} \bfseries EDICS Category: 3-BBND \end{center}
% \fi
%
% For peerreview papers, this IEEEtran command inserts a page break and
% creates the second title. It will be ignored for other modes.
\IEEEpeerreviewmaketitle



\section{Introduction}
% The very first letter is a 2 line initial drop letter followed
% by the rest of the first word in caps.
%
% form to use if the first word consists of a single letter:
% \IEEEPARstart{A}{demo} file is ....
%
% form to use if you need the single drop letter followed by
% normal text (unknown if ever used by IEEE):
% \IEEEPARstart{A}{}demo file is ....
%
% Some journals put the first two words in caps:
% \IEEEPARstart{T}{his demo} file is ....
%
% Here we have the typical use of a "T" for an initial drop letter
% and "HIS" in caps to complete the first word.

% \IEEEPARstart{I}{n} this paper we discuss a family of protocols that can be used for radio resource assignment in wireless networks.
% The problem of assigning resources in a distributed fashion is defined as a Decentralized Constraint Satisfaction (DCS) problem \cite{duffy2011dcs}.
% We consider a particular solver for this family of problems that works iteratively, in several rounds, to find a solution to the problem.

% The solver performs a stochastic search until it converges to a solution.
% You must have at least 2 lines in the paragraph with the drop letter
% (should never be an issue)

\IEEEPARstart{B}{itcoin} is a peer-to-peer electronic cash system \cite{nakamoto2008bpp} that maintains a public ledger with all transactions.
All the transactions need to be available to the peer-to-peer network that guarantees the security of the system.
Any full node of the network stores in a database all the transactions in the history of Bitcoin.
This database is typically referred to as the \emph{Blockchain}.

With traditional cash, there is no public record of all the transactions.
And the traditional banking system keeps the transactions of their customers private.
Therefore, the privacy that Bitcoin offers to its users is a matter study.
Ideally, any new payment system should offer privacy guarantees at least as good as traditional systems.

Bitcoin are sent to Bitcoin addresses, and both the Bitcoin community and previous research studies agree that address reuse is in general a bad practice.
It is recommended to generate a new address for each payment to be received.
This addresses that are used only ones are called disposable addresses.

In this paper we first review the basic elements of the Bitcoin system necessary for the subsequent discussion.
Then we discuss the necessity of avoiding address re-use to prevent unnecessary privacy leakage.
After that, we analyze the Blockchain to determine to which extend address reuse occurs in the Bitcoin community.
Disposable addresses do not offer total protection against privacy leakage and we detail the remaining risks as well as possible solutions.
Finally, we conclude.

\section{Address Reuse}

\subsection{Basic Bitcoin Elements}

Bitcoin is a protocol, a network, and an Internet currency unit.
We capitalize the word when we refer to either the protocol or the network.

Transactions are a fundamental element of Bicoin.
Payments require transactions, and these transactions are shared with the network and securely stored in the Blockchain.
Each transaction consumes some inputs and creates some outputs.
The inputs and outputs are worth bitcoins, and for a regular transaction to be valid the total value of the outputs must not exceed the total amount of the inputs.

The outputs of one transaction can be used as inputs of other transactions.
Critically, each output can be used only once.
The words consumed or spent are common synonyms of used in this context.
The idea is that each output can be spent only once just as we can spend our money only once.
The network does not accept transactions that try to spend outputs that have been spent before.

Another important element of Bitcoin are public/private asymmetric cryptographical keys.
The public key is hashed and coded with some redundancy into base58 addresses.
These addresses are alphanumeric chains that can be used to receive funds.
The outputs of a transaction can be send to an address, which is simply a convenient representation of a public key.

In order to spend an output, it is then necessary to offer proof of ownership of the address and, consequently, of the output.
This proof is the evidence of knowledge of the private key corresponding to the address.
A user willing to spend an output sent to a given address must provide the public address that hashes to the address and a valid signature.
The signature can be generated only by the owner of the private key.

There is no limit in Bitcoin regarding the number of outputs that can be send to a Bitcoin address.
The owner of the private key will be able to spend all of those outputs once.
Another particularity of bitcoin is that there is no practical limit of the number of addresses that can be generated.
There are $2^{160}$ possible addresses because they are generated using RIPE-MD160.
This makes it possible for users to generate new (disposable) addresses for each incoming payment.

A particular form of outputs which is relevant to the discussion later in this work are \emph{change} outputs.
A user willing to send a Bitcoin payment needs to combine in a transaction a number of inputs of value equal or larger than the desired payment.
If the value of the inputs is larger than the output, it is likely that the sending part does not want to loose the difference, also called change.
In order to keep the change, the payer creates a transaction with inputs exceeding the payment value and two outputs: the actual payment and the change.
The payment is sent to the payee and the change is sent to an address controlled by the payer.

It is a common and recommended practice that the amount of the inputs is slightly higher than the value of the outputs.
This small difference is called a \emph{fee} and it is kept by the peers that contributed to the security of the network.



\subsection{The Temptation}

The temptation exists of using Bitcoin addresses just like regular bank accounts numbers.
As addresses are used to receive payments, technically speaking only once is needed.
We can 


In most of the cases, outputs are linked to public key in an indirect way.
The outputs are destined to a particular address which is derived from a hash of a private key.
Only the owner of the associated private key is allowed to spend the output, after providing both the public key that hashes to the  and a


The public availability of this information has implications for the privacy of the users.
The public ledger consists of transactions that transfer funds from a set of input addresses to a set of output addresses.
As long as those addresses cannot be linked to their owners, privacy is preserved.
The linking of addresses to owners results in privacy leaks.
The possibilities of linking addresses to owners are multiplied when addresses are used to receive funds more than once.
In this work we privacy-leaking effects of address reuse and gather statistics of address reuse in the Bitcoin network.

In the following sections we offer an overview of the Bitcoin elements that are closely related to our discussion.
Many details and advanced features will be skipped to keep the discussion simple and focused.

\section{A Bitcoin Transaction}
A Bitcoin transaction transfers funds from one or more inputs to one or more outputs.
Inputs and outputs are linked to addresses, and normally those addresses are associated to a public key.
The owner of the private key owns the funds that are stored in the address.
The owner of the address can take a transaction that outputs funds to her address and use it as an input for another transaction that transfer those funds to some other address.
This newly created transaction must be signed with the private key of the input address (or addresses) in order to be accepted for the network.

The addresses that are linked to a single private key are alphanumeric (Base58) chains starting by 1 and around 30 characters long.
For convenience, in our examples we will use shorter addresses such as \texttt{1HB5X}.
Imagine that Bob and Chris generate a pair of private and public bitcoin keys each.
They derive their addresses from their public keys and obtain \texttt{1U8yx} and \texttt{1Y36v} respectively.

Ann sends one bitcoin to Bob's address \texttt{1U8yx}.
Bob wants to transfer that bitcoin to Chris' address \texttt{1Y36v}.
As Bob has the private key associated to the address \texttt{1U8yx}, he can create a transaction that has as an input Ann's transaction and outputs the funds to \texttt{1Y36v}.
Bob can provide a valid signature for address \texttt{1Bob} and broadcast the transaction to the Bitcoin peer-to-peer network.
This simple example is illustrated in Fig.~\ref{fig:simple-transaction}.

\begin{figure}[!t]
\centering
\includegraphics[width=\linewidth]{figures/simple-transaction}
\caption{The transaction on the left has been prepared by Ann to send funds to Bob. Bob uses the output of the left transaction to create a new transaction that sends the funds to Chris.}
\label{fig:simple-transaction}
\end{figure}

\section{Transactions with Multiple Inputs and Multiple Outputs}

The transaction presented in the previous section is the simplest in the sense that it has a single input and a single output.
It is often the case that Bitcoin transactions have multiple inputs and multiple outputs.
For example, if Bob wants to split the bitcoin that he has received from Alice in two parts, half for Chris and half for Dave, he can construct a transaction with two outputs of 0,5 bitcoins.
Similarly, if Bob receives two separate transactions of 0,5 bitcoins each, he can use those two outputs as inputs of a single transaction to send one bitcoin to Chris.

A recurrent situation in the Bitcoin network is that Bob has received a one bitcoin transaction and wants to send only half a bitcoin to Chris.
Bob can prepare a transaction in which he takes the one bitcoin transaction as an input, and splits it in two different outputs: one for Chris and one for himself.
The second output is usually referred to as \emph{the change} and it remains available for Bob to spend at a later time.

\section{Blockchain: The Public Ledger}

Every valid transaction is stored in a public ledger which is normally refferred to as the \emph{ Blockchain}.
When Ann sends a bitcoin to the address \texttt{1U8yx}, her transaction is included in the Blockchain.
Everyone in the peer-to-peer network knows that there is one output worth one bitcoin that is available for whoever has the private key associated to the \texttt{1U8yx} address.
When Bob creates a new transaction spending that output and it is accepted by the network and included in the Blockchain, the network knows that the one bitcoin output for \texttt{1U8yx} has been used and therefore it cannot be used again.

To prevent that an output can be spent twice it is necessary that all the Bitcoin network participants are aware of all existing transactions.
This makes it possible to track and verify which outputs are available for spending and therefore the amount of bitcoins available in each of the existing addresses.

The fact that all the transactions are public has some serious privacy implications.
The original bitcoin paper \cite{nakamoto2008bpp} points out that the security relies on keeping the addresses anonymous and that a new key pair should be used for each transaction.
In other words, that address reuse should be avoided.

\section{Address Reuse}

Address reuse is defined as the reception of more than one transaction in the same address.
It is possible to receive any number of transactions to the same address, and some users choose to do so.
A possible reason is the unawareness of the users of the privacy implications of address reuse.
In principle, creating a single address seems more convenient than generating a new one for every transaction.
This use is similar to that of a traditional bank account number.
The user needs to know a single address and single private key.
If all the payments are received to that address, the user can quickly check the balance of such address to know the amount of bitcoins that are available to her.

The downside is that that information is available to everyone.
If Alice, Bob and Chris use always the same address, they are leaking private information to each other.
As soon as Bob gives his address to Alive to receive a payment, Alice can know how many bitcoins Bob owns.
When Alice sends the payment to Bob, he learns her address and can also check how many bitcoins she owns.
If Alice sends a payment to Bob and a payment do Chris, and Bob sends a payment to Chris, the three parties know all the information about all the transactions, including those transactions in which they have not participated.
As Alice knows both Bob's and Chris' addresses, she can see examining the Blockchain that Bob send a payment to Chris.
Similarly, Bob knows that Chris received a payment from Alice and Chris knows that Bob received a payment from Alice.

\section{Wikileaks Privacy Leakage}

Wikileaks funding campaign is an example of address re-use.
At the time of this writing, Wikileak's donation webpage offers by default a re-used address.
It also offers the donors the possibility of generating a new (one-time) address by simply clicking a button.
The public address makes it for everyone to inspect all the details of the transactions involving that address.
A Blockchain explorer website (such as \texttt{blockchain.info}) can be used to browse all those details.
At the time of this writing, Wikileak's public address has received over 3,854 bitcoins in 2216 transactions.
The source addresses for each transaction are also public.

If Alice re-uses her address and sends a donation to Wikileaks and also sends some money to Bob, Bob can know that Alice sent money to Wikileaks as Alice address appears in both a donation to Wikileaks and in the transaction to Bob.
Address re-use makes it relatively easy to find out the balances and activities of the Bitcoin network participants.
Therefore, address re-use is discouraged in the original Bitcoin paper \cite{nakamoto2008bpp}.

\section{Evidence of Address Re-use in the Blockchain}

We use Obelisk and Libbitcoin to download and query the blockchain.
We count how many times each address appears as the output of a transaction.
Table~\ref{tab:address-reuse} shows some relevant statistics and Fig.~\ref{fig:address-reuse} presents an histogram showing addresses used hundred or less times.
Most of the addresses are used a single time, and a few address are used a large number of times.


\begin{table}[!t]
% increase table row spacing, adjust to taste
\renewcommand{\arraystretch}{1.3}
% if using array.sty, it might be a good idea to tweak the value of
% \extrarowheight as needed to properly center the text within the cells
\caption{Address Re-use Statistics}
\label{tab:address-reuse}
\centering
%% Some packages, such as MDW tools, offer better commands for making tables
%% than the plain LaTeX2e tabular which is used here.
\begin{tabular}{|cc|}
\hline
Mean & 3.18\\
Min & 1\\
25th perc. & 1 \\
50th perc. & 1 \\
75th perc. & 1 \\
Max & 1,238,931\\
Number of addresses & 12,963,199 \\
Number of uses & 41,244,997 \\
Addresses used once & 10,476,899 \\
Addressed used twice & 1,397,373\\
Used over 100 times & 25,004 \\
\hline
\end{tabular}
\end{table}

\begin{figure}[!t]
\centering
\includegraphics[width=\linewidth]{figures/address-reuse.eps}
\caption{Number of addresses for re-use factor from 1 to 100. Note the log scale.}
\label{fig:address-reuse}
\end{figure}

\section{Avoiding address re-use is not enough}

Even if addresses are used in a single occasion to receive bitcoins, it might be still possible to link different addresses to the same user.
The reason is that in most of the occasions a user does not have an output that is exactly the input that she needs.
If a user wants to spend 1 bitcoin and has two 0.5 bitcoin outputs that she can spend, she has to combine both of those outputs as inputs of the same transaction that spends, in total, one bitcoin.
A privacy attacker can infer that if the two inputs have been spent on the same transaction, they are likely to belong to the same user.

Another possibility of linking addresses is by means of \emph{change addresses}.A user might not find a combination of outputs that amounts exactly to the amount that she wants to spend in the transaction.
The solution is to create a transaction with two outputs. 
One of them is the actual payment while the other is sent to an address owned by the user that emits the payment.
This later address is a change address.
In \cite{meiklejohn2013fbc} some heuristics are provided to identify change addresses and use them to link addresses belonging to the same user.
The process is involved and several refinements are detailed  in \cite{meiklejohn2013fbc} to avoid false positives.
False negatives are still possible.

\section{Next Generation Privacy Enhancing Practices}

The privacy attacks described on the next section rely on the fact that all the inputs of the transaction belong to the same user.
Even though this is the current usual practice, it is by no means a restriction of bitcoin.
It is possible to create transactions with multiple inputs belonging to different users.
The technique of of creating join transactions among multiple users is called CoinJoin.

Bitcoins allows the construction of a transaction with an agreed set of outputs that is valid only when all the signatures needed for the inputs are provided.
Critically, the process requires that outputs are not changed.
Two or more users willing to create a join transaction must first agree on the payment addresses and amounts.
Then, each user provides the needed inputs (without signatures) and change addresses.
Finally, when the transaction is ready and only the signatures are missing, the users proceed to sign the transaction.
The user that has to provide the first signature has no incentive to cheat, as any change in the agreed transaction will be detected by the other participating users that will refuse to sign.
After the first user has signed, any change in the transaction will invalidate the first signature and therefore the whole transaction.
Therefore, trust is not needed by different users to create the transaction.

As there is uncertainty regarding the ownership of each of the inputs of the transaction, the amount of privacy leaked in CoinJoin transactions is greatly reduced.


\begin{figure}[!t]
\centering
\includegraphics[width=\linewidth]{figures/coinjoin.eps}
\caption{CoinJoin transaction of four users, five inputs, four payment outputs and four change outputs}
\label{fig:address-reuse}
\end{figure}

When a user is payed with a CoinJoin transaction, that user knows that one or more of the inputs belong to the payer, and likely one or more of the outputs belong to the payer.
However, the information of which are exactly the addresses belonging to the payer is not available.

%The stations deterministically choose their transmission slots after successful transmissions.
%If the transmission is not successful, the station randomly chooses its next transmission slot.
%The system eventually converges to collision-free operation in ideal channel conditions.



% if have a single appendix:
%\appendix[Proof of the Zonklar Equations]
% or
%\appendix  % for no appendix heading
% do not use \section anymore after \appendix, only \section*
% is possibly needed

% use appendices with more than one appendix
% then use \section to start each appendix
% you must declare a \section before using any
% \subsection or using \label (\appendices by itself
% starts a section numbered zero.)
%


%\appendices
%\section{Generalized Inclusion Exclusion Illustration}
%\label{app:incl-excl-thm}
%\label{app:theorem}
%We offer an example of the counting problem in (\ref{eq:Pij})  in which we are interested in computing the probability that exactly $delta$ among the $N$ stations succeed.
%
%\begin{figure}
%\psfrag{A1}[cc][cc]{$A_1$}
%\psfrag{A2}[cc][cc]{$A_2$}
%\psfrag{A3}[cc][cc]{$A_3$}
%\centering
%\includegraphics[height=3cm]{figures/counting}
%\caption{Illustration of the counting problem}
%\label{fig:counting}
%\end{figure}
%
%To illustrate the underlying idea we will use a simple example involving $N=3$ different stations choosing among $B=4$ different slots.
%We assume that the number of stations deterministically choosing its transmission slot is $d=0$ and we are interested in computing the probability that exactly $\delta=1$ station succeeds.
%
%There are a total of $B^N=4^3$ outcomes that are represented as black dots in Fig.~\ref{fig:counting}.
%Each outcome is equally likely with probability $1/64$.
%The figure also shows three events $A_1$, $A_2$ and $A_3$.
%The first event, $A_1$ represents the outcomes in which the first station succeeds.
%Similarly, the other two events, $A_2$ and $A_3$, represent the outcomes in which station 2 and 3 succeed, respectively.
%These sets are partially overlapping.
%Note also that it is impossible that two stations succeed while the third station collides.
%
%If we want to compute the probability that exactly one station succeeds, we have to count the outcomes that belong to either $A_1$, $A_2$ or $A_3$ and are not part of the intersections.
%This is $|A_1|+|A_2|+|A_3|-2(|A_1\cap A_2|+|A_1\cap A_3|+|A_2\cap A_3|)+3(|A_1\cap A_2\cap A_3|)$.
%In our paper we will use the intermediate values
% $S(1)=P(A_1)+P(A_2)+P(A_3)$,
% $S(2)=P(A_1\cap A_2)+P(A_1\cap A_3)+P(A_2\cap A_3)$, and $S(3)=P(A_1\cap A_2\cap A_3)$.
% The probability that exactly one station succeeds is finally computed  as $S(1)-2S(2)+3S(3)$.


% use section* for acknowledgement
\section*{Acknowledgment}


The authors would like to thank ...


% Can use something like this to put references on a page
% by themselves when using endfloat and the captionsoff option.
\ifCLASSOPTIONcaptionsoff
  \newpage
\fi



% trigger a \newpage just before the given reference
% number - used to balance the columns on the last page
% adjust value as needed - may need to be readjusted if
% the document is modified later
%\IEEEtriggeratref{8}
% The "triggered" command can be changed if desired:
%\IEEEtriggercmd{\enlargethispage{-5in}}

% references section

% can use a bibliography generated by BibTeX as a .bbl file
% BibTeX documentation can be easily obtained at:
% http://www.ctan.org/tex-archive/biblio/bibtex/contrib/doc/
% The IEEEtran BibTeX style support page is at:
% http://www.michaelshell.org/tex/ieeetran/bibtex/
\bibliographystyle{IEEEtran}
% argument is your BibTeX string definitions and bibliography database(s)
\bibliography{IEEEabrv,my_bib}
%
% <OR> manually copy in the resultant .bbl file
% set second argument of \begin to the number of references
% (used to reserve space for the reference number labels box)
%\begin{thebibliography}{1}

%\bibitem{IEEEhowto:kopka}
%H.~Kopka and P.~W. Daly, \emph{A Guide to \LaTeX}, 3rd~ed.\hskip 1em plus
%  0.5em minus 0.4em\relax Harlow, England: Addison-Wesley, 1999.

%\end{thebibliography}

% biography section
%
% If you have an EPS/PDF photo (graphicx package needed) extra braces are
% needed around the contents of the optional argument to biography to prevent
% the LaTeX parser from getting confused when it sees the complicated
% \includegraphics command within an optional argument. (You could create
% your own custom macro containing the \includegraphics command to make things
% simpler here.)
%\begin{biography}[{\includegraphics[width=1in,height=1.25in,clip,keepaspectratio]{mshell}}]{Michael Shell}
% or if you just want to reserve a space for a photo:

%\begin{IEEEbiography}{Michael Shell}
%Biography text here.
%\end{IEEEbiography}
%
%% if you will not have a photo at all:
%\begin{IEEEbiographynophoto}{John Doe}
%Biography text here.
%\end{IEEEbiographynophoto}
%
%% insert where needed to balance the two columns on the last page with
%% biographies
%%\newpage
%
%\begin{IEEEbiographynophoto}{Jane Doe}
%Biography text here.
%\end{IEEEbiographynophoto}

% You can push biographies down or up by placing
% a \vfill before or after them. The appropriate
% use of \vfill depends on what kind of text is
% on the last page and whether or not the columns
% are being equalized.

%\vfill

% Can be used to pull up biographies so that the bottom of the last one
% is flush with the other column.
%\enlargethispage{-5in}



% that's all folks
\end{document}

